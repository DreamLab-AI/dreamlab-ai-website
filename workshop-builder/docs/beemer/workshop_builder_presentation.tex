\documentclass{beamer}
\usepackage\[utf8]{inputenc}
\usepackage{graphicx} % For images, if any
\usepackage{amsmath}
\usepackage{amsfonts}
\usepackage{amssymb}
\usepackage{listings} % For code listings
\usepackage{tikz} % For more complex diagrams if needed, or use \includegraphics for Mermaid PNGs
\usetikzlibrary{shapes,arrows,positioning,calc}
\usepackage{hyperref} % For links

% Beamer Theme (Example: Madrid, choose one you like or is standard)
\usetheme{Madrid}

\title\[Workshop Builder]{Workshop Builder: OpenAI Codex Framework-Powered Curriculum Generation}
\author{The Workshop Builder Team}
\institute{DREAMLAB - AI}
\date{\today}

% Define colors (optional)
\definecolor{ocre}{RGB}{243,102,25}
\definecolor{mygray}{RGB}{128,128,128}
\definecolor{mygreen}{RGB}{0,128,0}
\definecolor{myblue}{RGB}{0,0,128}

% Listings style (optional)
\lstdefinestyle{mystyle}{
commentstyle=\color{mygreen},
keywordstyle=\color{blue},
numberstyle=\tiny\color{mygray},
stringstyle=\color{ocre},
basicstyle=\ttfamily\footnotesize,
breakatwhitespace=false,
breaklines=true,
captionpos=b,
keepspaces=true,
numbers=left,
numbersep=5pt,
showspaces=false,
showstringspaces=false,
showtabs=false,
tabsize=2
}
\lstset{style=mystyle}

\begin{document}

% --- Title Frame ---
\begin{frame}
\titlepage
\end{frame}

% --- Outline Frame (Table of Contents) ---
\begin{frame}{Outline}
\tableofcontents
\end{frame}

% --- Section 1: Introduction ---
\section{Introduction}

\begin{frame}{The Challenge: Scaling Knowledge Sharing}
\begin{itemize}
\item Creating high-quality, structured workshop materials is time-consuming.
\item Requires deep research, content structuring, formatting, and validation.
\item Manual workflows limit speed and consistency.
\end{itemize}
\end{frame}

\begin{frame}{Our Solution: Workshop Builder}
\textbf{Workshop Builder}: A Python CLI-based system that automates workshop module creation using AI.
\begin{itemize}
\item \textbf{Deep Research}: Gemini Flash 2.5 integration for comprehensive unstructured data gathering.
\item \textbf{Content Compilation}: \textbf{Actual} OpenAI Codex CLI integration for professional markdown generation.
\item \textbf{Multi-Agent Orchestration}: Coordinated agents (ResearchAgent, CompilerAgent, GitAgent) following Codex Framework.
\item \textbf{Professional Standards}: Enterprise-level output, comprehensive validation, robust error handling.
\item \textbf{Seamless GitHub Integration}: Automated PR creation with detailed descriptions and labeling.
\item \textbf{Docker & Security}: Hardened containerization with non-root execution, capability restrictions, read-only filesystem, resource limits.
\end{itemize}
\end{frame}

% --- Section 2: Core Architecture ---
\section{Core Architecture}

\begin{frame}{OpenAI Codex Framework Integration}
Workshop Builder implements a sophisticated multi-agent orchestration using the Codex Framework:
\begin{figure}
\centering
\begin{tikzpicture}\[node distance=2cm, auto,
block/.style={rectangle, draw, fill=blue!20, text width=10em, text centered, rounded corners, minimum height=2em},
codex/.style={rectangle, draw, fill=orange!30, text width=10em, text centered, rounded corners, minimum height=2em},
line/.style={draw, -latex'}]
\node \[block] (cli) {CLI (Python) - Codex Framework Entry};
\node \[codex, right of=cli, node distance=4.5cm] (orch) {Orchestrator};
\node \[block, above of=orch, node distance=2cm] (research) {ResearchAgent\small(Gemini Flash 2.5)};
\node \[codex, below of=orch, node distance=2cm] (compiler) {CompilerAgent\small(Codex CLI)};
\node \[block, right of=orch, node distance=4.5cm] (git) {GitAgent\small(GitHub API)};
\node \[block, right of=git, node distance=4cm] (pr) {Pull Request Ready};

```
    \path [line] (cli) -- node [midway,above] {Topic} (orch);
    \path [line] (orch) -- node [midway,left] {Deep Research} (research);
    \path [line] (research) -- node [midway,below] {Research Data} (orch);
    \path [line] (orch) -- node [midway,left] {Compile Prompt} (compiler);
    \path [line] (compiler) -- node [midway,below] {Module Files} (orch);
    \path [line] (orch) -- node [midway,above] {Publish} (git);
    \path [line] (git) -- (pr);
\end{tikzpicture}
\caption{Multi-Agent Architecture with Codex Framework}
\end{figure}
```

\end{frame}

\begin{frame}{Key Components}
\begin{itemize}
\item \textbf{CLI (`cli.py`)}: Accepts `--topic`, loads configuration, initializes Orchestrator.
\item \textbf{AppConfig (`config.py`)}: Loads `.env`, validates keys, configures Codex CLI integration, Docker paths.
\item \textbf{ResearchAgent}: Interfaces with Gemini Flash 2.5, generates unstructured research data.
\item \textbf{CompilerAgent}: Uses \texttt{codex} CLI for markdown generation, fallback to template-based when CLI unavailable, ensures professional formatting, Jinja2 templating for `manifest.json` & `README.md`.
\item \textbf{GitAgent}: Manages Git operations, branch creation, commit, push, and PR creation via PyGithub or simulated CLI.
\item \textbf{Docker Integration}: Secure container with non-root user, read-only filesystem, TMPFS, resource limits, healthchecks.
\end{itemize}
\end{frame}

% --- Section 3: Features & Capabilities ---
\section{Features & Capabilities}

\begin{frame}{Advanced Features}
\begin{itemize}
\item \textbf{Deep Research Integration}: Four targeted prompts to capture overview, applications, advanced concepts, learning path.
\item \textbf{Actual Codex CLI Integration}: Reliable `subprocess.run(['codex', ...])` enabling file creation and editing in `auto-edit` mode.
\item \textbf{Fallback Generation}: When CLI unavailable, generates minimal `00_introduction.md`, `01_core_concepts.md`, `manifest.json`, `README.md`.
\item \textbf{Professional Templates}: Jinja2 templates for `README.md` and `manifest.json` with metadata, learning objectives, prerequisites.
\item \textbf{Comprehensive Logging}: Structured logging via `structlog` or `loguru`, detailed DEBUG logs when `--verbose`.
\item \textbf{Error Handling}: Cascading recovery, retry logic (`MAX_RETRY_ATTEMPTS`), fallback generation, AGENTS.MD based fallback guidelines.
\item \textbf{Automated GitHub PRs}: Branch naming `workshop-XX-topic-slug`, descriptive PR body, labels (workshop, ai-generated, content).
\item \textbf{Docker & Security}: Non-root user (`workshop-user`), `--cap-drop=ALL`, `--read-only`, `--tmpfs`, `--memory` & `--cpus` limits, Trivy scanning in build.
\item \textbf{Beamer & Documentation}: Updated slides, detailed docs in `docs/`, modular structure for future enhancements.
\end{itemize}
\end{frame}

% --- Section 4: Docker & Security ---
\section{Docker & Security Enhancements}

\begin{frame}{Dockerfile Highlights}
\begin{itemize}
\item \textbf{Multi-Stage Build}: `builder` stage installs Python deps & Codex CLI, `production` stage adds non-root user and copies artifacts.
\item \textbf{Non-Root Execution}: Adds `workshop-user` (UID 1000), switches to non-root to run the application.
\item \textbf{Capability Restrictions}: `--cap-drop=ALL`, only `DAC_OVERRIDE` added for file operations.
\item \textbf{Read-Only Filesystem}: Root filesystem set as read-only, writable `tmpfs` mounted at `/tmp`.
\item \textbf{Resource Limits}: Memory: 2GB, CPU: 1.0, PIDs limit via `--pids-limit`.
\item \textbf{Security Scan}: Integrated Trivy scan during build (`build-docker.sh`).
\item \textbf{Healthcheck}: Verifies container can initialize `AppConfig` and find `.env`.
\end{itemize}
\end{frame}

\begin{frame}{Run Script Enhancements}
\begin{itemize}
\item \textbf{Enhanced `run-docker.sh`}: Parses flags `--no-limits`, `--memory`, `--cpu`, `--debug`.
\item \textbf{Environment Validation}: Ensures Docker running, image exists, `.env` file populated with real keys.
\item \textbf{Secure Mounts}: Mounts host `website/` directory into `/app`, replaces `.env` inside container via `--env-file`.
\item \textbf{Health & Monitoring}: Healthcheck via `docker inspect` and Python snippet for `AppConfig` load.
\item \textbf{Profiles}: `docker-compose` supports `dev` profile with debug flags and cache volume mounts.
\end{itemize}
\end{frame}

% --- Section 5: Workflow Demonstration ---
\section{Demonstration Workflow}

\begin{frame}{Demo: Generating a Workshop}
\begin{enumerate}
\item \textbf{Step 1: Configure `.env`}:
\begin{itemize}
\item Fill in `GEMINI_API_KEY`, `OPENAI_API_KEY`, `GITHUB_TOKEN`, `GITHUB_REPO_OWNER`, `GITHUB_REPO_NAME`.
\end{itemize}
\item \textbf{Step 2: Build Docker Image}:
\begin{lstlisting}\[language=bash]
./build-docker.sh
\end{lstlisting}
\item \textbf{Step 3: Run Workshop Builder}:
\begin{lstlisting}\[language=bash]
./run-docker.sh --topic "Understanding Kubernetes" --verbose
\end{lstlisting}
\item \textbf{Step 4: Observe Output}:
\begin{itemize}
\item ResearchAgent logs: files in `temp_research_data/`.
\item CompilerAgent logs: module created under `public/data/workshops/workshop-XX-understanding-kubernetes/`.
\item GitAgent logs: Branch `workshop-XX-understanding-kubernetes` created, pushed, PR URL displayed.
\end{itemize}
\item \textbf{Step 5: Review Pull Request} on GitHub.
\end{enumerate}
\end{frame}

% --- Section 6: Future Roadmap ---
\section{Future Roadmap}

\begin{frame}{Potential Enhancements}
\begin{itemize}
\item \textbf{ReviewAgent}: Automated review of generated content for accuracy and pedagogy.
\item \textbf{Interactive Mode}: Allow users to preview and edit sections between agent steps.
\item \textbf{SDK Integration}: Replace CLI calls with direct OpenAI SDK for finer control.
\item \textbf{Diagram Generation}: Integrate image generation for figures/diagrams within workshops.
\item \textbf{Web UI}: Web-based interface to manage workshop generation and review.
\item \textbf{Multi-Model Support}: Add support for additional LLMs (Claude, Llama) to diversify research and compilation.
\item \textbf{Enhanced Templates}: More Jinja2 templates for advanced metadata, quizzes, interactive elements.
\end{itemize}
\end{frame}

% --- Section 7: Conclusion ---
\section{Conclusion}

\begin{frame}{Summary}
\begin{itemize}
\item \textbf{Workshop Builder} automates professional workshop creation with real OpenAI Codex CLI and Gemini Flash 2.5.
\item \textbf{Multi-Agent Orchestration} ensures clear separation of concerns and robust error handling.
\item \textbf{Enterprise-Grade Output}: Structured markdown, Jinja2 templating, comprehensive logging, secure Docker.
\item \textbf{Seamless GitHub Integration}: Automated PR creation with detailed descriptions and labels.
\item \textbf{Security Best Practices}: Non-root container, capability restrictions, read-only filesystem, resource limits.
\end{itemize}
\vspace{1em}
\textbf{Key Innovation:} First fully integrated Codex CLI multi-agent orchestration system for automated curriculum generation.
\end{frame}

\begin{frame}
\centering
\Huge Thank You!
\vspace{1em}
\Large Questions?
\vspace{2em}
\normalsize
Visit our GitHub: \href{[https://github.com/your-org/website}{github.com/your-org/website}](https://github.com/your-org/website}{github.com/your-org/website})
\\
Documentation: \href{[https://website/docs}{website/docs}](https://website/docs}{website/docs})
\end{frame}

\end{document}
