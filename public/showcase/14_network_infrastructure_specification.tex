\chapter{Network Infrastructure Specification}

\section{Overview}

The network infrastructure forms the digital backbone of our immersive system, enabling seamless data flow between rendering clusters, tracking systems, storage arrays, and external collaborators with ultra-low latency and massive bandwidth.

\section{Network Architecture}

\subsection{Core Design Principles}

\begin{figure}[H]
\centering
\begin{tikzpicture}[scale=0.8]
    % Core switches
    \node[draw, fill=dreamlabSecondary!30, minimum width=2cm, minimum height=1cm] (core1) at (3,6) {Core SW1};
    \node[draw, fill=dreamlabSecondary!30, minimum width=2cm, minimum height=1cm] (core2) at (7,6) {Core SW2};
    \draw[very thick, dreamlabSecondary] (core1) -- (core2);

    % Spine switches
    \node[draw, fill=dreamlabPrimary!30, minimum width=2cm, minimum height=1cm] (spine1) at (2,4) {Spine 1};
    \node[draw, fill=dreamlabPrimary!30, minimum width=2cm, minimum height=1cm] (spine2) at (5,4) {Spine 2};
    \node[draw, fill=dreamlabPrimary!30, minimum width=2cm, minimum height=1cm] (spine3) at (8,4) {Spine 3};

    % Leaf switches
    \foreach \xpos/\xname/\label in {0/0/Compute,2.5/2p5/Storage,5/5/Tracking,7.5/7p5/Display,10/10/Mgmt} {
        \node[draw, fill=dreamlabAccent!30, minimum width=1.5cm, minimum height=0.8cm] (leaf\xname) at (\xpos,2) {\small\label};
    }

    % Connections
    \foreach \spine in {spine1,spine2,spine3} {
        \draw[thick] (core1) -- (\spine);
        \draw[thick] (core2) -- (\spine);
    }

    \foreach \leaf in {leaf0,leaf2p5,leaf5,leaf7p5,leaf10} {
        \foreach \spine in {spine1,spine2,spine3} {
            \draw[dreamlabDark] (\spine) -- (\leaf);
        }
    }

    % Devices
    \foreach \x in {0,2.5,5,7.5,10} {
        \draw[fill=dreamlabDark!20] (\x-0.3,0.5) rectangle (\x+0.3,1);
    }

    % External connection
    \draw[ultra thick, dreamlabPrimary, ->] (5,7) -- (5,8) node[above] {\textbf{Campus/Internet}};

    \node[below] at (5,-0.5) {\textbf{Non-Blocking Spine-Leaf Architecture}};
\end{tikzpicture}
\caption{High-performance network topology}
\end{figure}

\begin{requirement}{NET-001}{Full non-blocking architecture with line-rate forwarding}

\begin{requirement}{NET-002}{Sub-5 microsecond port-to-port latency}

\subsection{Network Segregation}

\begin{table}[H]
\centering
\begin{tabularx}{\textwidth}{@{}lXr@{}}
\toprule
\textbf{Network} & \textbf{Purpose} & \textbf{Bandwidth} \\
\midrule
Compute Fabric & GPU cluster interconnect & 400Gb/s \\
Storage Network & High-speed data access & 200Gb/s \\
Tracking Network & Camera and sensor data & 100Gb/s \\
Display Network & Video signal distribution & 100Gb/s \\
Management & Control and monitoring & 10Gb/s \\
Campus Uplink & External connectivity & 2×100Gb/s \\
\bottomrule
\end{tabularx}
\caption{Network segmentation strategy}
\end{table}

\section{Core Infrastructure}

\subsection{Switch Specifications}

\begin{requirement}{NET-003}{Enterprise-grade switches with deep buffers and RDMA support}

\subsubsection{Core/Spine Switches}
\begin{itemize}
    \item \textbf{Model}: Arista 7800R3 Series or equivalent
    \item \textbf{Ports}: 32× 400GbE QSFP-DD
    \item \textbf{Switching Capacity}: 25.6 Tbps
    \item \textbf{Latency}: <1 microsecond
    \item \textbf{Buffer}: 64GB shared memory
    \item \textbf{Features}: VXLAN, EVPN, segment routing
\end{itemize}

\subsubsection{Leaf Switches}
\begin{itemize}
    \item \textbf{Model}: Arista 7050X3 Series or equivalent
    \item \textbf{Ports}: 48× 25GbE + 8× 100GbE uplinks
    \item \textbf{Switching Capacity}: 3.2 Tbps
    \item \textbf{Features}: Cut-through switching, PTP support
\end{itemize}

\subsection{InfiniBand Option}

For ultimate compute performance, parallel InfiniBand deployment:

\begin{table}[H]
\centering
\begin{tabularx}{\textwidth}{@{}lX@{}}
\toprule
\textbf{Component} & \textbf{Specification} \\
\midrule
Technology & InfiniBand HDR200 (200Gb/s per port) \\
Switch & NVIDIA Quantum-2 QM8700 \\
Adapters & NVIDIA ConnectX-6 HDR \\
Topology & Fat-tree with 1:1 oversubscription \\
Features & GPUDirect RDMA, adaptive routing \\
\bottomrule
\end{tabularx}
\end{table}

\section{High-Performance Features}

\subsection{RDMA Implementation}

\begin{requirement}{NET-004}{Remote Direct Memory Access for compute and storage}

\begin{figure}[H]
\centering
\begin{tikzpicture}[scale=0.8]
    % Traditional vs RDMA
    \node[draw, fill=dreamlabPrimary!20] (cpu1) at (0,3) {CPU};
    \node[draw, fill=dreamlabAccent!20] (mem1) at (0,1) {Memory};
    \node[draw, fill=dreamlabPrimary!20] (cpu2) at (6,3) {CPU};
    \node[draw, fill=dreamlabAccent!20] (mem2) at (6,1) {Memory};

    % Traditional path
    \draw[thick, dreamlabSecondary] (mem1) -- (cpu1) -- (3,3) -- (cpu2) -- (mem2);
    \node[above] at (3,3) {\small Traditional};

    % RDMA path
    \draw[thick, dreamlabPrimary, dashed] (mem1) .. controls (3,0) .. (mem2);
    \node[below] at (3,0) {\small RDMA};

    \node[below] at (3,-1) {\textbf{RDMA Bypasses CPU for Direct Memory Access}};
\end{tikzpicture}
\caption{RDMA architecture benefit}
\end{figure}

\begin{itemize}
    \item \textbf{RoCE v2}: RDMA over Converged Ethernet
    \item \textbf{PFC}: Priority Flow Control for lossless transport
    \item \textbf{ECN}: Explicit Congestion Notification
    \item \textbf{DCQCN}: Data Center QCN for congestion control
\end{itemize}

\subsection{Time Synchronisation}

\begin{requirement}{NET-005}{Precision Time Protocol (PTP) for sub-microsecond synchronisation}

\begin{itemize}
    \item \textbf{Grandmaster Clock}: GPS-disciplined oscillator
    \item \textbf{Accuracy}: <100 nanoseconds across network
    \item \textbf{Boundary Clocks}: In all switches
    \item \textbf{Monitoring}: Continuous offset tracking
\end{itemize}

\section{Storage Network}

\subsection{Dedicated Storage Fabric}

\begin{table}[H]
\centering
\begin{tabularx}{\textwidth}{@{}lX@{}}
\toprule
\textbf{Feature} & \textbf{Implementation} \\
\midrule
Protocol & NVMe over Fabrics (NVMe-oF) \\
Transport & RoCE v2 or InfiniBand \\
Bandwidth & 40GB/s aggregate throughput \\
Latency & <100 microseconds to first byte \\
Redundancy & Dual-path with automatic failover \\
\bottomrule
\end{tabularx}
\end{table}

\subsection{Storage Area Network Details}

\begin{itemize}
    \item \textbf{Metadata Network}: Separate 25GbE for Lustre/BeeGFS
    \item \textbf{Object Storage}: 100GbE per storage node
    \item \textbf{Backup Network}: Isolated 10GbE for snapshots
    \item \textbf{Replication}: Dedicated link for DR site
\end{itemize}

\section{Camera and Sensor Networks}

\subsection{Tracking Network Architecture}

\begin{figure}[H]
\centering
\begin{tikzpicture}[scale=0.7]
    % PoE switches
    \node[draw, fill=dreamlabAccent!30, minimum width=3cm] (poe1) at (0,2) {PoE Switch 1};
    \node[draw, fill=dreamlabAccent!30, minimum width=3cm] (poe2) at (5,2) {PoE Switch 2};
    \node[draw, fill=dreamlabAccent!30, minimum width=3cm] (poe3) at (10,2) {PoE Switch 3};

    % Cameras
    \foreach \x in {-1,0,1} {
        \draw[fill=dreamlabPrimary!20] (\x,0) rectangle (\x+0.5,0.5);
        \draw[->] (\x+0.25,0.5) -- (\x+0.25,1.5);
    }
    \foreach \x in {4,5,6} {
        \draw[fill=dreamlabPrimary!20] (\x,0) rectangle (\x+0.5,0.5);
        \draw[->] (\x+0.25,0.5) -- (\x+0.25,1.5);
    }
    \foreach \x in {9,10,11} {
        \draw[fill=dreamlabPrimary!20] (\x,0) rectangle (\x+0.5,0.5);
        \draw[->] (\x+0.25,0.5) -- (\x+0.25,1.5);
    }

    % Aggregation
    \node[draw, fill=dreamlabSecondary!30, minimum width=4cm] (agg) at (5,4) {Aggregation Switch};
    \draw[thick] (poe1) -- (agg);
    \draw[thick] (poe2) -- (agg);
    \draw[thick] (poe3) -- (agg);

    % Processing
    \node[draw, fill=dreamlabPrimary!30] (proc) at (5,6) {Processing Cluster};
    \draw[ultra thick] (agg) -- (proc);

    \node[below] at (5,-0.5) {\textbf{Camera Network Topology}};
\end{tikzpicture}
\caption{Distributed PoE infrastructure for cameras}
\end{figure}

\begin{itemize}
    \item \textbf{PoE++ Support}: 90W per port for powered devices
    \item \textbf{10GigE Vision}: Camera data protocol support
    \item \textbf{Multicast}: Efficient distribution of tracking data
    \item \textbf{QoS}: Priority queuing for tracking packets
\end{itemize}

\section{External Connectivity}

\subsection{Campus Network Integration}

\begin{requirement}{NET-006}{Dual diverse 100GbE uplinks to campus core}

\begin{itemize}
    \item \textbf{Primary Path}: Direct fiber to data centre
    \item \textbf{Secondary Path}: Diverse route via alternate building
    \item \textbf{Routing}: BGP with campus AS
    \item \textbf{Filtering}: Stateful firewall at boundary
\end{itemize}

\subsection{Remote Collaboration Support}

\begin{table}[H]
\centering
\begin{tabularx}{\textwidth}{@{}lX@{}}
\toprule
\textbf{Service} & \textbf{Implementation} \\
\midrule
Video Streaming & WebRTC with TURN/STUN servers \\
Remote Rendering & NVIDIA CloudXR or similar \\
File Transfer & GridFTP for large datasets \\
API Access & REST/GraphQL via reverse proxy \\
VPN & IPSec for secure remote access \\
\bottomrule
\end{tabularx}
\end{table}

\section{Security Architecture}

\subsection{Network Security Layers}

\begin{figure}[H]
\centering
\begin{tikzpicture}[scale=0.8]
    % Zones
    \draw[thick, dreamlabSecondary] (0,0) rectangle (10,1.5);
    \node at (5,0.75) {\textbf{Public DMZ}};

    \draw[thick, dreamlabPrimary] (0,2) rectangle (10,3.5);
    \node at (5,2.75) {\textbf{Research Network}};

    \draw[thick, dreamlabAccent] (0,4) rectangle (10,5.5);
    \node at (5,4.75) {\textbf{Compute Core}};

    % Firewalls
    \draw[fill=red!30] (4.5,1.5) rectangle (5.5,2);
    \draw[fill=red!30] (4.5,3.5) rectangle (5.5,4);

    \node[right] at (10.5,0.75) {Web services};
    \node[right] at (10.5,2.75) {User access};
    \node[right] at (10.5,4.75) {Protected systems};

    \node[below] at (5,-0.5) {\textbf{Defence in Depth Security Model}};
\end{tikzpicture}
\caption{Security zone architecture}
\end{figure}

\subsection{Security Features}

\begin{itemize}
    \item \textbf{Micro-segmentation}: VXLAN-based isolation
    \item \textbf{Access Control}: 802.1X with RADIUS
    \item \textbf{Encryption}: MACsec on critical links
    \item \textbf{DDoS Protection}: Hardware-based mitigation
    \item \textbf{IDS/IPS}: Inline threat detection
\end{itemize}

\section{Quality of Service}

\subsection{Traffic Classification}

\begin{requirement}{NET-007}{Strict QoS prioritisation for real-time traffic}

\begin{table}[H]
\centering
\begin{tabularx}{\textwidth}{@{}lXr@{}}
\toprule
\textbf{Traffic Class} & \textbf{Applications} & \textbf{Priority} \\
\midrule
Real-time Control & Tracking, sync signals & Highest \\
Render Data & GPU cluster traffic & High \\
Storage I/O & File system operations & Medium \\
Management & Monitoring, logs & Low \\
\bottomrule
\end{tabularx}
\end{table}

\subsection{QoS Implementation}

\begin{itemize}
    \item \textbf{DSCP Marking}: At ingress points
    \item \textbf{Priority Queues}: 8 hardware queues
    \item \textbf{Rate Limiting}: Per-flow policers
    \item \textbf{Buffer Management}: Dynamic thresholds
\end{itemize}

\section{Management and Monitoring}

\subsection{Network Management System}

\begin{itemize}
    \item \textbf{DCIM}: Integrated with facility management
    \item \textbf{Configuration}: Ansible automation
    \item \textbf{Monitoring}: Prometheus + Grafana
    \item \textbf{Flow Analysis}: sFlow/NetFlow collection
    \item \textbf{Performance}: Real-time latency tracking
\end{itemize}

\subsection{Monitoring Metrics}

\begin{requirement}{NET-008}{Comprehensive monitoring with 1-second granularity}

\begin{itemize}
    \item Port utilisation and error rates
    \item Latency histograms per path
    \item Buffer utilisation
    \item PTP clock offset
    \item RDMA congestion events
    \item Security event correlation
\end{itemize}

\section{Cabling Infrastructure}

\subsection{Fiber Optic Specifications}

\begin{table}[H]
\centering
\begin{tabularx}{\textwidth}{@{}lX@{}}
\toprule
\textbf{Type} & \textbf{Application} \\
\midrule
OS2 Single-mode & Long runs, campus uplinks \\
OM5 Multi-mode & Spine-leaf connections \\
MPO/MTP Trunks & High-density backbone \\
AOC Cables & Top-of-rack connections \\
\bottomrule
\end{tabularx}
\end{table}

\subsection{Structured Cabling}

\begin{itemize}
    \item \textbf{Cable Management}: Overhead ladder trays
    \item \textbf{Labelling}: Automated documentation system
    \item \textbf{Testing}: OTDR certification for all fiber
    \item \textbf{Redundancy}: Diverse physical paths
\end{itemize}

\section{Resilience and Redundancy}

\subsection{High Availability Features}

\begin{itemize}
    \item \textbf{Switch Redundancy}: MLAG/VARP configurations
    \item \textbf{Link Redundancy}: LAG with min-links
    \item \textbf{Power}: Dual PSU from separate PDUs
    \item \textbf{Cooling}: N+1 fan modules
\end{itemize}

\subsection{Failure Scenarios}

\begin{requirement}{NET-009}{Zero downtime for single component failure}

\begin{itemize}
    \item Switch failure: <50ms convergence
    \item Link failure: Sub-second LACP failover
    \item Power failure: Seamless PSU transition
    \item Software failure: Hitless upgrades
\end{itemize}

\section{Future Expansion}

\subsection{Scalability Provisions}

\begin{itemize}
    \item 50\% port capacity reserved for growth
    \item 400GbE/800GbE upgrade path
    \item Additional spine switches pre-cabled
    \item Power and cooling headroom
\end{itemize}

\subsection{Emerging Technologies}

\begin{itemize}
    \item \textbf{P4 Programmability}: Custom packet processing
    \item \textbf{Network Slicing}: Guaranteed SLA per application
    \item \textbf{AI/ML Integration}: Predictive failure detection
    \item \textbf{Quantum Networking}: Research provisions
\end{itemize}

\begin{center}
\begin{tikzpicture}
\node[rectangle, draw=dreamlabPrimary, fill=dreamlabLight!20, text width=14cm, inner sep=15pt, rounded corners] {
\centering
\textbf{\large\color{dreamlabPrimary}Network Infrastructure Summary}\\[0.5cm]
Our network architecture delivers the ultra-low latency and massive bandwidth required for real-time immersive experiences. With non-blocking switching, RDMA support, and comprehensive redundancy, the infrastructure ensures that data flows seamlessly between all system components while maintaining the sub-millisecond timing precision essential for multi-user immersion.
};
\end{tikzpicture}
\end{center}